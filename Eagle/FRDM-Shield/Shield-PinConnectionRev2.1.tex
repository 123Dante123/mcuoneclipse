\documentclass[10pt,a4paper]{article}
\usepackage[utf8]{inputenc}
\usepackage{amsmath}
\usepackage{amsfonts}
\usepackage{amssymb}
\usepackage{makeidx}
\author{Andrin Tuor}
\title{FRDM-Board Pin Connection Rev 2.1}

\newcommand{\textoverline}[1]{$\overline{\mbox{#1}}$}

\begin{document}

\section*{IEEE 802.15.4 Board Pin Connection Rev 2.1}

\hspace*{-3cm}
\begin{tabular}{|c|c|c|c|c|}
\hline \textbf{Signal Name} & \textbf{KL25Z Pin} & \textbf{Arduino} & \textbf{Description} & \textbf{I/O Header} \\ 
\hline MISO & PTD7/SPI1\_MISO &  & SPI Master input, slave output & J2 19 \\ 
\hline MOSI & PTD6/SPI1\_MOSI &  & SPI Master output, slave input & J2 17 \\ 
\hline SPICLK & PTD5/SPI1\_SCK & D9 & SPI Serial Clock & J2 04 \\ 
\hline \textoverline{SS} & PTD4/SPI1\_PCS0 & D2 & SPI Slave select (active low) & J1 06 \\
\hline \textoverline{RESET} & PTA13 & D8 & Allows MCU to reset the MC1320x (active low) & J2 02 \\ 
\hline \textoverline{IRQ} & PTD2 & D11 & \textoverline{IRQ} line from MC1320x (active low) & J2 08 \\ 
\hline \textoverline{RXTXEN} & PTA17 &  & Initiate transceiver functions (active low) & J2 11 \\ 
\hline \textoverline{ATTN} & PTC7 &  & Wake up from low power modes (active low) & J1 01 \\  
\hline 
\end{tabular} 

\vspace{2cm}
\flushleft
\textbf{Pin change compared to previous revisions:}
\flushleft
\emph{\textoverline{IRQ} pin removed from PTE31 (J2 13) and now connected to PTD2 (J2 08)!}

\end{document}